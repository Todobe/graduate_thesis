\cleardoublepage
\chapternonum{摘要}

随着移动设备和社交网络的快速发展,广告序列推荐成为一个重要而具有挑战性的问题。现有研究侧重于用户对广告的偏好,但却未充分关注到广告间正向关系、用户选择性接受广告和社交网络影响力等因素对序列实际价值的影响。针对这些问题,本文从广告平台的角度提出了两个创新性算法,旨在为用户推荐更加合适的序列广告,从而最大化平台的收益。

针对广告间正向关系和用户选择性接受广告的问题,本文提出了鲁棒序列网络子模最大化算法。该算法引入网络子模函数来增强广告间正向关系的重要性,并采用两段式贪心策略来提高序列的鲁棒性,以在用户拒绝部分广告的最坏情况下实现收益最大化。本文严格证明了该算法在不同情况下的近似比,并分析了其理论性能。

进一步,结合信息在用户社交网络中的传播,本文提出了多轮社交广告序列影响最大化问题,首次将广告网络和用户社交网络同时纳入模型,可以综合评估广告之间的正向关系和用户网络信息传播对广告推荐带来的影响。本文设计了基于边贪心的算法框架,并提出了多轮反向影响力采样方法来提高计算效率。为解决目标函数非子模性的问题,引入了三明治方法来保证算法的近似比。此外,本文还拓展了动态算法以适应动态场景。

最后,在真实数据集上进行了广告推荐、链路预测和多轮广告序列推荐实验。结果表明,本文提出的算法在评价指标上均优于现有最优方法,展示了优秀的有效性和鲁棒性。这些实验不仅验证了算法的实际应用价值,也为广告序列推荐领域提供了新的解决方案和研究方向。


\textbf{关键词}:广告推荐,子模最大化,影响力最大化,社交网络。


\cleardoublepage
\chapternonum{Abstract}

With the rapid development of mobile devices and social networks, advertising sequence recommendation has become an important and challenging issue. Existing research primarily focuses on user preferences for advertisements, but fails to fully consider the impact of factors such as positive relationships between advertisements, user selective acceptance of ads, and social network influence on the actual value of the sequence. In response to these issues, this dissertation proposes two innovative algorithms from the perspective of advertising platforms, aiming to recommend more suitable sequence advertisements to users in order to maximize platform revenue.

To address the issues of positive relationships between advertisements and user selective acceptance of ads, this dissertation introduces the Robust Sequence Networked Submodular Maximization algorithm. The algorithm incorporates a net-submodular function to enhance the importance of positive relationships between advertisements and utilizes a two-stage greedy strategy to improve the robustness of the sequence, thereby maximizing revenue even in the worst-case scenario where users reject some ads. This dissertation rigorously prove the approximation ratio of this algorithm under different conditions and analyze its theoretical performance.


Furthermore, in conjunction with the propagation of information in user social networks, this dissertation propose the Multi-Round Social Advertising Sequence Influence Maximization problem. For the first time, both the advertising network and user social network are incorporated into the model, allowing for a comprehensive consideration of the impact of positive relationships between advertisements and user network information propagation on ad recommendations. This dissertation design an edge-based greedy algorithm framework and propose a multi-round reverse influence sampling method to improve computational efficiency. To address the non-submodularity of the objective function, the sandwich method is introduced to guarantee the approximation ratio of the algorithm. Additionally, the algorithm is extended to adapt to dynamic scenarios.

Finally, experiments on real datasets are conducted for tasks including product advertisement recommendation, link  prediction and multi-round sequence advertisement recommendation. The results show that proposed algorithms outperform existing methods across evaluation metrics, demonstrating excellent effectiveness and robustness. These experiments not only validate the practical value of the algorithms but also provide new solutions and research directions for the field of advertising sequence recommendation.

\textbf{Keywords}:Advertising Recommendation, Submodular Maximization, Influence Maximization, Social Network.